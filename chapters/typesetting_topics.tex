%%
\chapter{Typesetting Topics}
\label{chapter:typesetting}

This chapter will go through some of the \tex features you'll probably want to use at some point.
So far, {\em most} of the things I usually do with \tex can be made to work for eBook outputs,
but there are lots of commands and options that don't work and you need to know which
variants to use.

\section{References and Citations}

One of the easy things that `just works' most of the time in \latex is references. For example,
the \tex source for this chapter starts with:

\begin{verbatim}
\chapter{Typesetting Topics}
\label{chapter:typesetting}
\end{verbatim}

Then if I want to refer to the chapter, I write \smalltt{Chapter \textbackslash ref\{chapter:typesetting\}} which
gets rendered as `Chapter \ref{chapter:typesetting}'. Adding \smalltt{label} commands for tables and figures
works in the same way.

This isn't rocket science, but if you've ever tried
renumbering by-hand, you know how valuable this is! Similarly, bibliographic references such as
\smalltt{\textbackslash cite\{knuth1984texbook\}} work correctly, giving a citation looking like \cite{knuth1984texbook}.
So far \textsc{BibTeX} has worked fine for this book.

%%
\section{Graphics}

Nearly all eBooks have graphics in them somewhere, even if just for a cover page. 
The \smalltt{graphicx} package works well for this, with an \smalltt{fbox} to add a frame.
For example, the cover image for the title page of this book is included using the command:

\begin{verbatim}
\fbox{\includegraphics[width=0.8\linewidth]{images/cover.png}}
\end{verbatim}

Typically for webpages and eBooks, the size of images is expected to vary with the page size and settings,
so using a context-sensitive width like a proportion of the linewidth is more appropriate than a fixed-width
declaration like `10cm'. Many eReaders enable users to click on images to see them in more detail, which helps.

One frequent problem is that HTML typesetting using \smalltt{htlatex} or a similar process doesn't always preserve the
aspect ratio of your images. For example, the same setting may be used for width and height, making all images square.
The problem is solved by adding a bounding box file using the \smalltt{extractbb} command, which is typically included
with \tex distributions. For this book, this step is included in the project's \smalltt{build.sh} script, at least for \smalltt{.jpg} and
\smalltt{.png} files, and it can easily be extended to more filetypes. Or you can run \smalltt{extractbb -x \$IMAGE} on these files
yourself.

Images are often put in the \latex \smalltt{figure} environment, which can include captions and a label for references.
For example, the map in Figure \ref{fig:sea_map} is created using the commands:

\begin{verbatim}
\begin{figure}
\begin{center}
\caption{A map showing countries in Southeast Asia, 
         made using the \smalltt{pilmaps} library}
  \label{fig:sea_map}
  \includegraphics[width=\linewidth]{images/sea_countries.png}
\end{center}
\end{figure}
\end{verbatim}
  
\begin{figure}
\begin{center}
\caption{A map showing countries in Southeast Asia, 
         made using the \smalltt{pilmaps} library}
  \label{fig:sea_map}
  \includegraphics[width=\linewidth]{images/sea_countries.png}
\end{center}
\end{figure}

From here, the figure can be referenced using the command 
\smalltt{\textbackslash ref\{fig:sea\_map\}}. (And if you happen to be looking for a free mapping tool in python that
supports low-level, hands-on rendering control, feel free to try {\small \url{https://github.com/dwiddows/pilmaps}}.) 
So far I haven't found an effective way of controlling the placement of figures --- for example, directives
like \smalltt{\textbackslash begin[ht]\{figure\}} don't affect the HTML or the ePub output, and I haven't got
wraparound text to work.

 %%
\section{Tables}

Basic tables typeset just fine. For example, the following \latex gives the output in Table \ref{tab:artist_works}.

\begin{verbatim}
\begin{table}
  \centering
  \begin{tabular}{|c|l|}
    \hline
    \textbf{Artist} & \textbf{Great Works} \\
    \hline
    Leonardo da Vinci & The Mona Lisa \\
    Charlie Watts & Satisfaction \\
    \hline
  \end{tabular}
  \caption{Contributions to modern civilization}
  \label{tab:artist_works}
\end{table}
\end{verbatim}

\begin{table}
 \begin{center}
  \begin{tabular}{|c|l|}
    \hline
    \textbf{Artist} & \textbf{Great Works} \\
    \hline
    Leonardo da Vinci & The Mona Lisa \\
    Charlie Watts & Satisfaction \\
    \hline
  \end{tabular}
  \caption{Contributions to modern civilization}
  \label{tab:artist_works}
 \end{center}
\end{table}

With standard \latex it is easy to make tables that are too big for pages or that typeset poorly for other of reasons.
I expect this can be even more of a problem with small-screen eBooks, so you'll want to design any tables accordingly
and check output carefully.

%%
\section{Equations}

One of the benefits of \latex is that it's easy to typeset mathematical notation like formulae and equations.

So far these look to work well in HTML and eBook formats. The commands below give the subsequent
renderings of Euler's formula and Fourier series:

\begin{verbatim}
\[ e^{ix} = \cos x + i \sin x \]
\[ f(x) = \sum_{0}^{\infty}a_k \sin(kx) + b_k \cos(kx). \]
\end{verbatim}

\[ e^{i\pi} = -1 \]
\[ f(x) = \sum_{0}^{\infty}a_k \sin(kx) + b_k \cos(kx). \]

Array environments for typesetting rows and columns in equations also work, for example:

\[
u = \left( \begin{array}{c} 1 \\ 0 \\ -2 \end{array} \right) \qquad
v = \left( \begin{array}{c} 2 \\ -1 \\ 3 \end{array} \right) \qquad
u^T v = 2 + 0 - 6 = -4 \qquad
u v^T = \left( \begin{array}{ccc} 2 & -1 & 3 \\ 0 & 0 & 0 \\ -4 & 2 & -6 \end{array} \right)
\]

This set of equations may become typeset in a smaller font than those above, to fit them
horizontally on a small screen. If these start to get too small, consider breaking lines up
when typesetting mathematics for eBooks.

%%
\section{Contents and Index}

The table of contents should get typeset entirely automatically if you use this template.

The story of how traditional indexes and concordances influenced the design of the inverted
indexes used by search engines is fascinating \citep[Ch 1]{witten1999gigabytes}.
Since this story is so advanced by now, I'm not adding an explicit index section ---
to look something up, use the Search button! If you want to make a paper
version of your book as well as an eBook, you may want to reconsider this.

If you do want to make a paper book as well, it may be worth trying a
more sophisticated template such as the one that comes with Clemens
Lode's book on using \latex for books and eBooks \citep{lode2019better}.
This includes conditional compilation so that different commands and even sections are used
for the PDF version that leads to a paper book and the HTML version that leads to an eBook.