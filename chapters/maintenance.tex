%%%
\chapter{Maintenance and Troubleshooting}

We're nearly done with this short template book --- at least with the first version.
But as with most topics to do with software and electronic information, things will keep changing.
Bugs will hopefully get fixed, new bugs will arise, different devices will support different formats,
standards may change --- for example, it may become
possible to load \smalltt{.epub} files directly onto an Amazon Kindle, rather than having to use the
\smalltt{.mobi} format solely for this purpose. This last chapter will include a few suggestions on how
to ask questions and report changes.

%%
\section{Report Bugs on Github}

The open source github repository for this project is {\small \url{https://github.com/dwiddows/ebookbook/}}.
If something recommended in this book doesn't work for you, and if you have to change or add extra
commands to make it work, please report this as an issue there. The github project wiki can be used for
keeping instructions up-to-date and adding links to more resources. This is likely to be much more effective
the reporting problems on sites where the book is sold.
If you want to review the book itself, post a review. But if you want to report a problem and get
help, please use the project github site.

When there are major developments or enough new things to include, I will add these to future
editions of this book (if there are any!). But by design, books are meant to be relatively stable, whereas
project websites and wikis are designed to have immediate updates. If a new `edition' comes out too often,
we can lose track of what to refer to.

So tl;dr: please leave actual reviews on marketplace sites, but report bugs on github. Not just because
I want to avoid a bunch of frustrated bug reports as book reviews, but also because if
bugs are reported on the github project, they're more likely to be addressed and fixed!

%%
\section{That's All For Now}

This was always intended to be a short book, just enough to provide a working template, to demonstrate
a few typesetting features, and to make the book whose end you're just reaching.

I hope it's given you some confidence to get started, and in particular, I hope the structure works for you.
Clone the project, install a few dependencies, try it out, and ideally within a short time you won't be thinking
much about this book --- it will already be becoming {\em your} book.

Good luck and enjoy your writing!

\begin{flushright}
  {\em Dominic Widdows \\ September 2021}
\end{flushright}




